\chapter{Wstęp}

Postęp technologiczny w zakresie sztucznej inteligencji oraz przetwarzania języka naturalnego (ang. \textit{Natural Language Processing}, NLP) sprawia, że interakcja człowieka z komputerem staje się coraz bardziej intuicyjna. Jak zauważył Alan Turing już w 1950 roku\footnote{A.~M. Turing. \textit{Computing Machinery and Intelligence}. \textit{Mind}, Vol.~59, No.~236, pp.~433--460, 1950.}: „Maszyna może być uznana za inteligentną, jeśli jej odpowiedzi są nieodróżnialne od odpowiedzi człowieka”. Dziś wizja ta znajduje odzwierciedlenie w systemach zdolnych do interpretacji ludzkich wypowiedzi i podejmowania działań w oparciu o ich znaczenie.

Inspiracją do podjęcia niniejszego tematu była obserwacja potrzeb środowiska hodowców koni – grupy zawodowej, która mimo dużego doświadczenia praktycznego często cechuje się niskim stopniem zinformatyzowania. Pomysł stworzenia aplikacji zrodził się z rozmowy z jednym z hodowców, wujkiem jednego z członków zespołu, który zwrócił uwagę na brak narzędzi umożliwiających wygodne zarządzanie stadniną. Zidentyfikowany problem oraz szybki rozwój technologii NLP doprowadziły do koncepcji stworzenia aplikacji sterowanej językiem naturalnym.

Celem projektu stało się opracowanie systemu webowego umożliwiającego zarządzanie stadniną koni oraz obsługę aplikacji poprzez komendy w języku naturalnym. Tego rodzaju rozwiązanie może istotnie obniżyć próg wejścia dla użytkowników nieposiadających kompetencji technicznych, a jednocześnie stanowić przykład praktycznego wykorzystania nowoczesnych modeli językowych w rzeczywistym środowisku.

Zakres pracy obejmuje analizę problemu, zaprojektowanie i implementację aplikacji webowej \textit{Moje Konie}, stworzenie klasyfikatora intencji użytkownika, a także opracowanie metody automatycznego generowania zapytań REST na podstawie wypowiedzi naturalnych. Zakres czasowy pracy obejmował pełny cykl wytwarzania oprogramowania – od analizy wymagań, poprzez implementację i integrację, aż po testy oraz ocenę skuteczności zastosowanych metod.

W pracy przyjęto hipotezę, że możliwe jest skuteczne sterowanie aplikacją webową poprzez wypowiedzi użytkownika w języku naturalnym, przy użyciu współczesnych modeli językowych (LLM), takich jak Gemini~2.5~Pro. Zakłada się również, że odpowiednio zaprojektowany klasyfikator intencji i moduł interpretacji poleceń umożliwią poprawne odwzorowanie wypowiedzi na konkretne żądania aplikacyjne.

Źródła literaturowe wykorzystane w pracy obejmują zarówno klasyczne opracowania z zakresu przetwarzania języka naturalnego, jak i najnowsze publikacje dotyczące dużych modeli językowych, architektury REST oraz projektowania aplikacji webowych. Ważnym odniesieniem stały się również dokumentacje narzędzi i bibliotek programistycznych wykorzystanych w projekcie, takich jak React, Node.js, PostgreSQL oraz Python.

Podczas realizacji projektu zespół napotkał na wyzwania związane z integracją komponentów aplikacji oraz dostosowaniem modeli NLP do specyficznej dziedziny, jaką jest hodowla koni. Kluczowe było także zapewnienie odpowiedniej komunikacji pomiędzy warstwą frontendową i serwerową, a także stworzenie środowiska umożliwiającego eksperymentalną ocenę skuteczności modeli.

\begin{quote}
Celem pracy jest opracowanie aplikacji webowej \textit{Moje Konie} wykorzystującej przetwarzanie języka naturalnego do sterowania funkcjami systemu oraz analiza skuteczności zastosowanych metod klasyfikacji intencji i mapowania wypowiedzi użytkownika na żądania REST.
\end{quote}
\begin{quote}
Struktura pracy jest następująca:
\begin{description}
    \item[Rozdział~2] przedstawia przegląd literatury dotyczącej przetwarzania języka naturalnego oraz dużych modeli językowych.
    \item[Rozdział~3] opisuje projekt i implementację serwerowej części aplikacji do zarządzania stadniną koni.
    \item[Rozdział~4] zawiera opracowanie i ocenę eksperymentalną klasyfikatora intencji użytkownika.
    \item[Rozdział~5] opisuje metodę wypełniania żądań REST na podstawie wypowiedzi użytkownika przy użyciu modelu językowego LLM.
    \item[Rozdział~6] prezentuje integrację opracowanych modułów w aplikacji \textit{Moje Konie}.
    \item[Rozdział~7] zawiera podsumowanie oraz wnioski z przeprowadzonej pracy.
\end{description}
\end{quote}
\begin{quote}
Tomasz Pawłowski opracował architekturę systemu oraz był odpowiedzialny za implementację części serwerowej aplikacji. Jakub Buler pełnił funkcję głównego programisty, odpowiadając za implementację i integrację aplikacji webowej. Adam Detmer, jako kierownik projektu, zajmował się analizą biznesową, zarządzaniem zadaniami zespołu oraz opracowaniem klasyfikatora intencji użytkownika. Jakub Kamieniarz odpowiadał za przygotowanie modułu generowania żądań REST z wykorzystaniem modelu językowego Gemini~2.5~Pro.
\end{quote}
\bibliographystyle{plain}
\begin{thebibliography}{1}
\bibitem{turing1950} A.~M. Turing. \textit{Computing Machinery and Intelligence}. \textit{Mind}, Vol.~59, No.~236, pp.~433--460, 1950.
\end{thebibliography}
