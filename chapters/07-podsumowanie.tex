\chapter{Podsumowanie i wnioski}

Niniejsza praca przedstawia proces projektowania i implementacji zaawansowanego systemu zarządzania stajnią \definicja{Moje Konie}, łączącego klasyczne podejście inżynierskie z nowoczesnymi metodami przetwarzania języka naturalnego. Fundamentem technologicznym rozwiązania stała się architektura SPA/PWA, oparta na rygorystycznym rozdzieleniu warstw frontendowej i backendowej. Kluczowym osiągnięciem projektowym było wdrożenie modelu, zapewniającego izolację danych między hodowlami, oraz uzyskanie pełnego bezpieczeństwa typów na każdym etapie przepływu informacji dzięki wykorzystaniu bibliotek \textit{Zod} i \textit{DrizzleORM}. 

Stabilność środowiska produkcyjnego zagwarantowano poprzez nowoczesną warstwę wdrożeniową, wykorzystującą potoki CI/CD, konteneryzację oraz paradygmat \textit{Infrastruktury jako kod}. Zastosowanie mechanizmu \textit{Cosign} do podpisywania obrazów oraz izolacja usług w sieciach Docker pozwoliły na stworzenie bezpiecznego i skalowalnego ekosystemu, zdolnego do bezawaryjnej obsługi procesów administracyjnych przy wykorzystaniu frameworka \textit{Hono} oraz zasobów \textit{Google Cloud Storage}.

Warstwa prezentacji systemu stanowi kompleksową realizację nowoczesnego interfejsu webowego, w którym biblioteka \textit{React.js} i język \textit{TypeScript} zapewniają wysoką wydajność oraz płynność renderowania. Dzięki podejściu \textit{Mobile First} i frameworkowi \textit{Tailwind CSS}, aplikacja jest w pełni zoptymalizowana pod kątem specyficznych warunków pracy terenowej. Najistotniejszym elementem interfejsu stał się \definicja{Asystent NLP}, który transformuje pasywne wprowadzanie danych w aktywny dialog.

Istotną część pracy poświęcono optymalizacji mechanizmu odwzorowania języka naturalnego na strukturalne żądania REST. Badania eksperymentalne wykazały, że o ile proste podejście \textit{Zero-Shot} prowadzi do licznych halucynacji, o tyle wdrożenie strategii \textit{Few-Shot Prompting} z dynamicznym wstrzykiwaniem kontekstu pozwala osiągnąć wysoką precyzję oraz efektywność. Przełomem inżynierskim w tym zakresie okazała się autorska, wieloetapowa orkiestracja zapytań. Rozdzielenie etapu rozpoznawania zwierzęcia od etapu tworzenia gotowego zapytania pozwoliło uniknąć błędów w zapisach medycznych i zapewniło pełną spójność danych w bazie.

Dalsza optymalizacja modułu \definicja{Asystenta NLP} doprowadziła do zastąpienia kosztownej architektury monolitycznej modelem hybrydowym. Wykorzystanie klasyfikatora intencji opartego na osadzeniach wektorowych \texttt{multilingual-e5-large-instruct} umożliwiło zastosowanie techniki usuwania zbędnego kontekstu (\english{context pruning}), co znacząco zredukowało szum informacyjny i koszty operacyjne. Ewolucja od pojedynczego algorytmu k-NN do metody zespołowej (\english{Ensemble Learning}), opartej na głosowaniu wielu klasyfikatorów, pozwoliła na osiągnięcie wysokiej stabilności decyzyjnej. System wykazał wysoką odporność na błędy gramatyczne i skrótowość myślową, skutecznie interpretując branżowe słownictwo nawet w trudnych przypadkach brzegowych.

Realizacja projektu była dla autorów niezwykle satysfakcjonującym procesem badawczym, stanowiącym cenną lekcję pokory wobec złożoności języka naturalnego. Zaobserwowanie momentu, w którym teoretyczne założenia o możliwościach modeli LLM przekształciły się w praktyczne, stabilne narzędzie realnie ułatwiające pracę w stadninie, przyniosło ogromną radość. Praca nad systemem pozwoliła na dogłębne poznanie pełnego cyklu życia aplikacji -- od projektowania baz danych, przez bezpieczne potoki CI/CD, aż po zaawansowaną inżynierię promptów. Rozwiązanie problemów integralności danych medycznych udowodniło, że kluczem do sukcesu jest synergia między klasycznym rygorem programistycznym a elastycznością sztucznej inteligencji. Projekt ten utwierdził autorów w przekonaniu, że technologia powinna być przede wszystkim „przezroczystym” wsparciem dla człowieka, a połączenie pasji informatycznej z rozwiązywaniem realnych problemów dziedzinowych stanowi najbardziej inspirującą ścieżkę rozwoju zawodowego. % TODO - lekcja pokory może być? XD