\chapter{Architektura systemu}

\section{Założenia projektowe}

Zaprojektowany system spełnia następujące wymagania funkcjonalne:

\begin{itemize}
    \item Użytkownik może się uwierzytelnić na ekranie logowania.
    \item Po pomyślnym uwierzytelnieniu użytkownik jest autoryzowany do podejmowania działań na obiektach należących do wskazanej hodowli.
    \item Użytkownika ma możliwość zmiany składnika uwierzytelniającego (hasła).
    \item Po podaniu niepoprawnych danych uwierzytelniających przez użytkownika system wyświetla komunikat błędu.
    \item Użytkownik po zalogowaniu może zobaczyć wszystkie swoje konie (przypisane dla danej hodowli), 
    \item Strona pokazująca listę koni udostępnia wyszukiwarkę, za pomocą której można filtrować wyświetlanie konie.
    \item Strona z listą koni umożliwia przejście do strony z informacjami szczegółowymi danego konia.
    \item System umożliwia podejrzenie zdjęcia konia i jego powiększenie.
    \item System pozwala na dodanie wielu zdjęć do jednego konia, przełączanie się pomiędzy nimi, a także na wybór ulubionego zdjęcia.
    \item Po dodaniu pierwszego zdjęcia do konia jest ono ustalane jako ulubione.
    \item Użytkownik ma możliwość dodania, edycji oraz usunięcia koni z systemu.
    \item System pozwala na nawigację pomiędzy modułami funkcjonalności takimi jak: Konie, Weterynarze, Kowale, Wydarzenia w stajni, Asystent, Ustawienia  
    \item System umożliwia dodawanie wydarzeń dla wielu koni, czyli podkuć i zdarzeń profilaktycznych.
    \item Spośród zdarzeń profilaktycznych użytkownik ma wybór pomiędzy: dentystą, podaniem witamin, szczepieniem, odrobaczaniem.
    \item W celu dodania zdarzenia profilaktycznego użytkownik musi wybrać jego typ, uzupełnić wymagane pola, a także wybrać listę koni, do których dane zdarzenie ma zastosowanie.
    \item W miejscach, gdzie użytkownik ma do wyboru konie z listy, dostępny jest również przycisk do zaznaczenia wszystkich.
    \item Użytkownik ma możliwość dodania do tabeli weterynarza, z którym współpracuje.
    \item System podpowiada użytkownikowi weterynarzy, gdy użytkownik dodaje nowe wydarzenie wymagające podania weterynarza.
    \item Użytkownik ma możliwość dodania do tabeli kowala, z którym współpracuje.
    \item System podpowiada użytkownikowi kowali, gdy użytkownik dodaje nowe podkucie.
    \item Użytkownik może zobaczyć informacje takie jak: numer przyżyciowy, numer chipa, rocznik urodzenia, rodzaj, płeć, data przybycia do stajni, data odejścia ze stajni, a także najnowsze aktywne wydarzenia: podkucie, odrobaczanie, podanie suplementów, szczepienie, dentystę odnośnie wybranego konia na ekranie ze szczegółami. 
    \item Użytkownik może przejść z ekranu szczegółów danego konia bezpośrednio do ostatnich wydarzeń z nim związanych.
    \item Użytkownik może wygenerować raport zawierający informacje o danym koniu oraz wybrać podzbiór wydarzeń i ich zakres czasu (według daty zdarzenia).
    \item TODO: opis zdarzeń profilaktycznych, jakie pola, co wymagane, jak obliczane
    \item System wysyła powiadomienia o zbliżającym się upływie ważności podkucia lub zdarzenia profilaktycznego do użytkowników.
    \item Użytkownik ma możliwość ustawić preferencje wysyłanych powiadomień (zbiorcze dla wszystkich koni) dla każdego typu wydarzenia z osobna, z jakim wyprzedzeniem ma przyjść informacja o wygaśnięciu ważności (ile dni wyprzedzenia, o której godzinie), a także sposób transportu wiadomości (email, \st{push})
\end{itemize}


Wymagania pozafunkcjonalne:

\begin{itemize}
    \item Logowanie użytkownika odbywa się poprzez email oraz hasło.
    \item Do zmiany hasła wymagane jest podanie starego hasła oraz dwukrotne podanie nowego hasła.
    \item Treść hasła jest domyślnie maskowana na ekranie logowania, użytkownik może ją podejrzeć podczas wpisywania.
    \item przechowywanie danych autoryzacji w sposób bezpieczny, wyłącznie hashe bcrypt haseł,
    \item Lista koni jest wyświetlana jako karty ze zdjęciem (wybranym jako ulubione) i nazwą konia. W formacie przeznaczonym na większe ekrany są to 3 kolumny, natomiast na urządzeniach mobilnych jest to 1 kolumna.
    \item Po zalogowaniu wyświetlana jest lista koni.
    \item Filtrowanie koni na liście odbywa się po nazwie za pomocą wyszukania podciągu znaków frazy.
    \item Ekran pokazujący zdjęcie konia jest pokazywany po kliknięciu na zdjęcie i jest powiększany do pełnego rozmiaru karty.
    \item Dodawanie i edycja informacji dotyczących koni odbywa się poprzez dedykowany formularz walidujący podstawową poprawność pól (właściwie tylko sprawdza czy wymagane pola są uzupełnione)
    \item Operacje usunięcia wymagają potwierdzenia.
    \item Użytkownik po wybraniu widoku dodawania zdarzeń profilaktycznych ma do wyboru 5 kafelków, każdy odpowiada akcji dodania odpowiedniego zdarzenia.
    \item Użytkownik przełącza się pomiędzy wieloma zdjęciami danego konia za pomocą strzałek na ekranie szczegółów.
    \item Pod ostatnimi wydarzeniami znajduje sie panel do dodawania wydarzeń, gdzie do wybrania jest dodanie wydarzenia związanego z rozrodami/chorobami/leczeniami/zdarzeniami profilaktycznymi /podkucia
    \item Niewielki koszt i prostota administracyjna utrzymania w dłuższej perspektywie
    \item Obsługa 400 łącznych użytkowników, w tym 100 jednoczesnych sesji.
    \item Wysoki poziom bezpieczeństwa, modularnośc i izolacja komponentów systemu
    \item Intuicyjny i wygodny interfejs graficzny, działający w przeglądarce internetowej oraz jako aplikacja mobilna
\end{itemize}

Coś czego nie mamy (aby się nie walnąć):

\begin{itemize}
    \item resetowanie hasła przez użytkownika, tylko admin z dostępem do bazy danych (lub ogólnie całej instancji API)
    \item rejestracja użytkowników przez użytkownika, j.w.
    \item rejestracja hodowli przez użytkownika, j.w.
    \item ustawienie złożoności hasła,
    \item integracja z zewnętrznymi systemami uwierzytelniającymi
    \item rate limiting prób logowania, ban po zbyt wielu próbach
    \item utrzymywanie sesji logowania, sesja trwa 15 minut i się nie odświeża (coś było implementowane, ale nie działa)
    \item wysyłanie powiadomień push (próbowałem dodać, nie udało mi się)
\end{itemize}

Błędy, które nie mogą nam umknąć:
\begin{itemize}
    \item Sesja logowania nie utrzymuje się
    \item Przycisk wyloguj nie działa - sesja dalej pozostaje
    \item Można kliknąć na zdjęcie na ekranie z listą koni, aby je powiększyć zamiast przejść do szczegółów konia
    \item Można powiększyć domyślne zdjęcie - jego brak
    \item Na mniejszym ekranie (u mnie na laptopie 13 cali) nagłówek ma dziwne odstępy
    \item Miejscami zdarzenia są zapisywane jako wydarzenia i na odwrót
    \item Dodawanie zdjęć nie działa - błąd z json payload
    \item Koń po dodaniu ma jakiś pusty wpis ze zdjęciem, przez co cały czas się kręci, ale nic nie pokazuje
    \item Przekierowanie z / do /konie nie działa
    \item Czerwone komunikaty błędu na ciemnozielonym gradiencie tła są słabo widoczne (mają zbyt niski kontrast)
    \item Tworzenie raportu dla konia bez wydarzeń nie działa (przykład koń aaa)
    \item W ustawieniach powiadomień są wciąż opcje "Oba" i "Push", które nie działają w oczekiwany sposób
\end{itemize}


\section{Opis architektury aplikacji}

Ze względu na specyfikę aplikacji umożliwiającą edycję danych dotyczących danej hodowli koni wielu użytkownikom, w szczególności korzystających w większości z urządzeń mobilnych oraz sporadycznie z komputerów osobistych, zdecydowano się na formę systemu webowego z interfejsem przeglądarkowym interfejsem użytkownika. Aplikacje internetowe pozwalają na wykorzystanie bogatego i dojrzałego zaplecza technologicznego do wyświetlania stron internetowych, znacząco upraszcza aspekt dystrybucji oprogramowania oraz współbieżnego dostępu do danych.

Korzystając z technologii progresywnych aplikacji internetowych (ang. Progressive Web App, PWA) oraz praktyki responsywnego projektowania stron internetowych (ang. responsive web design, RWD) zaimplementowano spójny interfejs użytkownika dla platform mobilnych. Do implementacji technologii progresywnych aplikacji internetowych wymagane jest, aby interfejs graficzny użytkownika był skomponowany w zdecydowanej większości ze statycznych dokumentów hipertekstowych HTML (HyperText Markup Language), arkuszy stylów CSS (Cascading Style Sheets) oraz kodu w języku programowania JavaScript. Statyczne pliki w podanym kontekście oznaczają podleganie nieczęstym zmianom poprzez pobranie zaktualizowanej wersji z serwera - wyłącznie między oddzielnymi wydaniami aplikacji i nigdy w celu wyświetlenia nowych danych. Interaktywna prezentacja danych jest możliwa natomiast poprzez zmianę struktury wyświetlanego dokumentu hipertekstowego wykonywaną po stronie klienta z użyciem kodu JavaScript w przeglądarce. Wymienione działanie jest wykonywane poprzez interfejs programowania Document Object Model (DOM, pl. \textit{Model Obiektowy Dokumentu}) pozwalający na częściową lub pełną modyfikację dowolnego węzła (np. paragrafu tekstu, separatora, kontrolek, obiektu multimedialnego) w hierachicznej strukturze drzewa dokumentu HTML. Interfejs ten jest w praktyce wykorzystywany przez bibliotekę React, która nieco modyfikuje model z dokumentów HTML i skryptów JavaScript na komponenty wielokrotnego użytku (zawierające wymagane fragmenty dokumentu HTML i funkcje JavaScript) budujące deklaratywne widoki (strony internetowe). Komponenty można powiązać z danymi, czyli zaprojektować ich wygląd w celu prezentacji danych zgodnie z ich schematem oraz skonfigurować ich źródło, które będzie automatycznie odświeżane dążąc do zadeklarowanego stanu. Komponenty można w sobie zagnieżdżać, odtwarzając te same aspekty hierachicznej struktury, a ich zmiany są śledzone poprzez użycie wewnętrznej dla tej biblioteki kopii Modelu Obiektowego Dokumentu nazywanego Virtual (pl. Wirtualnego) DOM.

Opisane podejście nazywa się Aplikacją jednostronicową (ang. \textit{Single-page application, SPA}), co wynika głównie ze sposobu aktualizowania interfejsu na podstawie zmian w danych poprzez kod po stronie klienta i przesyłania statycznych dokumentów HTML, pomijając jednocześnie aspekt użycia wielu stron. W przypadku prezentowanej pracy rama projektowa została dostarczona przez bibliotekę React Router, która dostarcza strategię renderowania do Aplikacji jednostronicowej, dostarczając ponadto .

Dla zachowania kompatybilności z technologią progresywnych aplikacji internetowych oraz modularności zdecydowano się na rozdzielenie warstwy przetwarzania logiki biznesowej z dostępem do danych, z którą użytkownik nie wchodzi w bezpośrednią interakcję - nazywaną "back end", od warstwy bezpośrednio obsługującej interfejs użytkownika i wizualną prezentację danych - "front end".

\subsection{React Router}

Do implementacji routingu po stronie klienta wykorzystano framework React Router. Zapewnia on:

\begin{itemize}
    \item lekkość i prostotę konfiguracji,
    \item obsługę renderowania po stronie klienta,
    \item wsparcie dla React 18 oraz React 19,
    \item możliwość zarówno minimalnego, jak i kompleksowego wykorzystania w aplikacji.
\end{itemize}

React Router korzysta z bibliotek i narzędzi: React, TypeScript, Vite oraz Nginx. Projekt uzupełniają również VitePWA, Zod oraz Tailwind CSS.

\subsection{React}

React pełni rolę biblioteki odpowiedzialnej za budowę interfejsu. Został wybrany ze względu na:

\begin{itemize}
    \item bardzo dużą popularność i wsparcie społeczności,
    \item prostotę i deklaratywny model programowania,
    \item komponentową strukturę umożliwiającą skalowanie aplikacji,
    \item możliwość ponownego wykorzystania kodu,
    \item znajomość technologii przez zespół projektowy.
\end{itemize}

React umożliwia tworzenie złożonych interfejsów przy użyciu komponentów zarządzających własnym stanem, z możliwością integracji z TypeScript i narzędziami budującymi.

\subsection{TypeScript}

TypeScript został wykorzystany w celu rozszerzenia języka JavaScript o statyczny system typów, co zapewnia:

\begin{itemize}
    \item poprawę jakości kodu,
    \item redukcję błędów wykonania,
    \item lepsze narzędzia developerskie (np. IntelliSense),
    \item kompatybilność z nowoczesnymi bibliotekami frontendowymi.
\end{itemize}

\subsection{Vite}

Vite jest narzędziem developerskim i bundlerem zaprojektowanym z myślą o optymalnej wydajności.

\begin{itemize}
    \item Bardzo szybki serwer deweloperski oparty o moduły ES.
    \item Obsługa natychmiastowego przeładowania (HMR).
    \item Integracja z Rollup na potrzeby kompilacji produkcyjnej.
    \item Wsparcie dla PWA poprzez plugin VitePWA.
\end{itemize}

\subsection{Zod}

Zod jest biblioteką walidacji danych typu TypeScript-first. Zapewnia:

\begin{itemize}
    \item spójne walidowanie danych po obu stronach aplikacji,
    \item bogaty system błędów i prosty interfejs API,
    \item integrację z narzędziami backendowymi i ORM,
    \item generowanie schematów JSON Schema oraz OpenAPI.
\end{itemize}

\subsection{Tailwind CSS}

Tailwind CSS został użyty do budowy warstwy prezentacji poprzez zestaw niskopoziomowych klas użytkowych, co umożliwia szybkie prototypowanie oraz utrzymanie spójnej stylistyki interfejsu.

\subsection{Nginx}

W środowisku produkcyjnym frontend jest udostępniany przez serwer Nginx. Wybrano go ze względu na:

\begin{itemize}
    \item wysoką wydajność,
    \item niewielki narzut zasobów,
    \item bogatą funkcjonalność mimo niewielkiego rozmiaru,
    \item sprawdzone działanie i dojrzałość technologii.
\end{itemize}

\section{Backend}

Backend realizuje logikę biznesową aplikacji, odpowiada za dostęp do bazy danych, uwierzytelnianie użytkowników oraz komunikację z usługami zewnętrznymi.

\subsection{Hono}

Hono jest lekkim frameworkiem webowym zgodnym ze standardami Web Standards, działającym na wielu środowiskach wykonawczych. Najważniejsze cechy:

\begin{itemize}
    \item bardzo wysoka wydajność routera,
    \item brak zewnętrznych zależności,
    \item obsługa środowisk Node.js, Deno, Bun, Cloudflare Workers i innych,
    \item szeroki zestaw wbudowanych middleware’ów,
    \item pełne wsparcie dla TypeScript.
\end{itemize}

Hono pełni rolę warstwy REST API oraz kanału RPC umożliwiającego komunikację z frontendem z wykorzystaniem wspólnych schematów danych.

\subsection{Zod w backendzie}

Podobnie jak we frontendzie, Zod odpowiada za walidację danych wejściowych, generowanie schematów OpenAPI oraz integrację z DrizzleORM, zapewniając pełną spójność typów.

\subsection{DrizzleORM}

DrizzleORM pełni funkcję warstwy dostępu do danych. Zapewnia:

\begin{itemize}
    \item definiowanie schematów tabel zgodnych z TypeScript i Zod,
    \item API wzorowane na SQL, zapewniające niski narzut,
    \item bezpieczeństwo przed SQL injection,
    \item narzędzia do migracji bazy danych,
    \item interoperacyjność z PostgreSQL.
\end{itemize}

\subsection{Biblioteki dodatkowe}

Backend wykorzystuje również:

\begin{itemize}
    \item NodeMailer — wysyłanie wiadomości e-mail przez SMTP Google,
    \item Google Cloud Storage — przechowywanie zdjęć i generowanie podpisanych odnośników,
    \item Google Generative AI — komunikacja z modelem Gemini w celu przetwarzania zapytań języka naturalnego,
    \item node-cron — harmonogramowanie zadań cyklicznych,
    \item Winston — logowanie zdarzeń,
    \item bcrypt — bezpieczne hashowanie haseł użytkowników,
    \item dotenv — ładowanie zmiennych środowiskowych.
\end{itemize}

\section{Przechowywanie danych}

\subsection{PostgreSQL}

Głównym silnikiem bazy danych jest PostgreSQL. Wybrano go z uwagi na:

\begin{itemize}
    \item wysoką stabilność i niezawodność,
    \item zgodność z DrizzleORM,
    \item duży ekosystem narzędzi administracyjnych,
    \item dobrą wydajność w scenariuszach transakcyjnych.
\end{itemize}

Dane zapisywane są na dedykowanej partycji dyskowej, co ogranicza ryzyko problemów związanych z brakiem miejsca oraz zapewnia izolację operacji I/O.

\subsection{Kopie zapasowe}

Do wykonywania kopii zapasowych zastosowano:

\begin{itemize}
    \item borgbackup — deduplikacja i kompresja danych,
    \item borgmatic — automatyzacja harmonogramów, rotacja backupów, integracja z narzędziami administracyjnymi.
\end{itemize}

Zgodnie z założeniami projektu backupowane są wyłącznie dane z bazy, a kod aplikacji może zostać przywrócony z repozytorium.

\section{Moduł klasyfikatora}

Moduł klasyfikatora odpowiada za analizę zapytań języka naturalnego oraz przypisywanie ich do odpowiednich operacji API. Został wdrożony jako oddzielna mikro-usługa z uwagi na odmienny zestaw zależności.

Wykorzystuje:

\begin{itemize}
    \item sentence-transformers — generowanie osadzeń tekstowych,
    \item scikit-learn — podział danych oraz metody ewaluacyjne,
    \item numpy i torch — obliczenia tensorowe,
    \item FastAPI — udostępnienie punktów końcowych REST,
    \item Pydantic — walidacja danych wejściowych,
    \item uv — zarządzanie środowiskiem wykonawczym.
\end{itemize}

\section{Przepływ ruchu sieciowego}

Przepływ zapytań od użytkownika przebiega następująco:

\begin{enumerate}
    \item Użytkownik zapytuje DNS o domenę systemu, otrzymując adres serwerów Cloudflare, które pełnią funkcję WAF.
    \item Zapytanie trafia do infrastruktury Google Cloud Platform, gdzie jest filtrowane przez firewall.
    \item Ruch kierowany jest do instancji wirtualnej maszyny, w której Traefik pełni rolę reverse proxy.
    \item Traefik rozdziela ruch:
    \begin{itemize}
        \item ścieżka ``/'' — frontend (Nginx),
        \item ``/klasyfikator'' — moduł klasyfikatora,
        \item ``/api'' — backend aplikacji.
    \end{itemize}
\end{enumerate}

Backend po uwierzytelnieniu użytkownika za pomocą tokenów JWT realizuje żądania REST, mapując je na zapytania SQL przy użyciu DrizzleORM. W przypadku danych multimedialnych generowane są podpisane odnośniki do zasobów Cloud Storage. Dodatkowo backend komunikuje się z agentem Gemini LLM w celu analizy intencji użytkownika.

\section{Powtarzalność i izolacja środowiska uruchomieniowego (Docker)}

Komponenty aplikacji działają w izolowanych kontenerach Docker, co zapewnia powtarzalność środowiska oraz minimalizację zależności systemowych.

\textbf{TODO:} Opisać szczegółowo:
\begin{itemize}
    \item charakterystykę kontenerów i ich zalety,
    \item model warstw obrazu i wpływ na czas wdrażania,
    \item izolację zasobów oraz minimalne obrazy distroless,
    \item proces budowania obrazów w monorepo.
\end{itemize}

\section{Definicja infrastruktury jako kod (Terraform i Ansible)}

\textbf{TODO:} Rozszerzyć opis:
\begin{itemize}
    \item Terraform jako narzędzie Day~0 (provisioning VM, dysków, bucketów, firewalli),
    \item Ansible jako narzędzie Day~1--2 (konfiguracja systemu, Docker Engine, borgmatic),
    \item modułowość i powtarzalność procesów provisioningowych.
\end{itemize}

\section{Automatyzacja CI/CD}

Proces ciągłej integracji i wdrażania został oparty o GitHub Actions i Ansible.

\textbf{TODO:} Dopisać:
\begin{itemize}
    \item opis potoku kompilacji TypeScript,
    \item budowanie i publikację obrazów Docker,
    \item proces aktualizacji aplikacji na serwerze (pobranie obrazów, restart usług),
    \item model pracy z monorepo, pull requesty i kontrola jakości kodu.
\end{itemize}




