\chapter{Architektura systemu}

\section{Założenia projektowe}

System \definicja{Moje Konie} został zaprojektowany z myślą o zapewnieniu pełnej kontroli nad zarządzaniem stajnią oraz umożliwieniu użytkownikom wygodnego monitorowania stanu i historii koni w hodowli. Podstawowym mechanizmem zabezpieczającym dostęp jest proces logowania, w którym użytkownik uwierzytelnia się przy użyciu adresu e-mail oraz hasła. Po poprawnym uwierzytelnieniu użytkownik uzyskuje autoryzację do wykonywania działań na obiektach przypisanych do jego hodowli. System zapewnia możliwość zmiany hasła przez użytkownika po podaniu aktualnego hasła i dwukrotnym wprowadzeniu nowego hasła. Treść hasła jest domyślnie maskowana, a użytkownik ma możliwość wyświetlenia tekstu jawnego podczas wpisywania. W przypadku podania nieprawidłowych danych uwierzytelniających system wyświetla odpowiedni komunikat błędu.

Po zalogowaniu użytkownik otrzymuje dostęp do przeglądania pełnej listy swoich koni, zawierających zdjęcie oraz nazwę konia. Wyświetlanie listy jest responsywne: na większych ekranach stosowane są trzy kolumny, natomiast na urządzeniach mobilnych tylko jedna. Lista koni wyposażona jest w wyszukiwarkę, pozwalającą na filtrowanie po nazwie poprzez wyszukanie fragmentu tekstu. Każda karta konia umożliwia przejście do ekranu ze szczegółowymi informacjami, gdzie użytkownik może zobaczyć numer przyżyciowy, numer chipa, rocznik urodzenia, rodzaj, płeć, datę przybycia i odejścia ze stajni, a także najnowsze aktywne wydarzenia związane z danym koniem, takie jak podkucia, szczepienia, odrobaczanie, podanie suplementów czy wizyty dentystyczne.

System pozwala na zarządzanie zdjęciami koni –- użytkownik może dodawać wiele zdjęć do jednego konia, a przełączanie między zdjęciami odbywa się za pomocą strzałek na ekranie szczegółów. Operacje dodawania i edycji koni odbywają się poprzez dedykowane formularze walidujące wymagane pola, natomiast usuwanie wymaga potwierdzenia ze strony użytkownika.

Na poziomie nawigacji system udostępnia moduły funkcjonalne obejmujące: zarządzanie końmi, weterynarzami, kowalami, wydarzeniami w stajni, asystentem oraz ustawieniami. Przy dodawaniu zdarzenia użytkownik wybiera jego typ, uzupełnia wymagane pola oraz wskazuje listę koni, do których zdarzenie ma zastosowanie. W celu wygodnej selekcji dostępny jest również przycisk pozwalający zaznaczyć wszystkie konie w danej stadninie. System podpowiada użytkownikowi powiązanych weterynarzy lub kowali przy dodawaniu zdarzeń wymagających ich udziału. Pod ostatnimi wydarzeniami znajduje się panel umożliwiający szybkie dodanie nowych wydarzeń, odpowiednio pogrupowanych według kategorii.

Dodatkowo użytkownik ma możliwość generowania raportów dla poszczególnych koni, w tym wyboru podzbioru wydarzeń oraz określenia zakresu czasowego według daty zdarzenia. System wysyła powiadomienia o zbliżającym się upływie ważności podkucia lub zdarzenia profilaktycznego. Użytkownik może określić preferencje powiadomień, w tym czas wyprzedzenia (w dniach i godzinach) oraz sposób doręczenia wiadomości (np. e-mail).

Projektując system, zwrócono szczególną uwagę na bezpieczeństwo, modularność oraz skalowalność. Dane autoryzacyjne przechowywane są w formie bezpiecznych hashy przy użyciu algorytmu \definicja{bcrypt}. System jest w stanie obsłużyć łącznie 400 użytkowników, w tym do 100 jednoczesnych sesji. Interfejs użytkownika został zaprojektowany w sposób intuicyjny i wygodny, działający zarówno w przeglądarce internetowej, jak i na urządzeniach mobilnych. Koszt utrzymania systemu jest minimalny, a administracja systemem prosta, co zwiększa długoterminową trwałość i niezawodność rozwiązania.

Wszystkie te cechy łącznie pozwalają na zapewnienie wysokiego poziomu bezpieczeństwa danych, przejrzystej organizacji informacji oraz wygodnej obsługi codziennych czynności w hodowli koni.


% TODO: Coś czego nie mamy (aby się nie walnąć):

% \begin{itemize}
%     \item resetowanie hasła przez użytkownika, tylko admin z dostępem do bazy danych (lub ogólnie całej instancji API)
%     \item rejestracja użytkowników przez użytkownika, j.w.
%     \item rejestracja hodowli przez użytkownika, j.w.
%     \item ustawienie złożoności hasła,
%     \item integracja z zewnętrznymi systemami uwierzytelniającymi
%     \item rate limiting prób logowania, ban po zbyt wielu próbach
%     \item utrzymywanie sesji logowania, sesja trwa 15 minut i się nie odświeża (coś było implementowane, ale nie działa)
%     \item wysyłanie powiadomień push (próbowałem dodać, nie udało mi się)
% \end{itemize}

% TODO: Błędy, które nie mogą nam umknąć:
% \begin{itemize}
%     \item Można powiększyć domyślne zdjęcie - jego brak
%     \item Miejscami zdarzenia są zapisywane jako wydarzenia i na odwrót
%     \item Koń po dodaniu ma jakiś pusty wpis ze zdjęciem, przez co cały czas się kręci, ale nic nie pokazuje
%     \item Przekierowanie z / do /konie nie działa
%     \item Czerwone komunikaty błędu na ciemnozielonym gradiencie tła są słabo widoczne (mają zbyt niski kontrast)
%     \item Tworzenie raportu dla konia bez wydarzeń nie działa (przykład koń aaa)
%     \item W ustawieniach powiadomień są wciąż opcje "Oba" i "Push", które nie działają w oczekiwany sposób
%     \item Wydarzenia są pobierane WSZYSTKIE NA RAZ a następnie stronnicowane i sortowane po stronie klienta
%     \item Walidacja w wielu miejscach odbywa się tylko po stronie klienta, np. nr telefonu
%     \item Chat po dodaniu nie ma odświeżanego stanu bazy danych np. dodanie konia, dodania szczepienia dla tego konia
%     \item Wydarzenia są pobierane WSZYSTKIE NA RAZ a następnie stronnicowane i sortowane po stronie klienta
%     \item Walidacja w wielu miejscach odbywa się tylko po stronie klienta, np. nr telefonu
%     \item Chat po dodaniu nie ma odświeżanego stanu bazy danych np. dodanie konia, dodania szczepienia dla tego konia
% \end{itemize}


\section{Opis architektury aplikacji}

\begin{figure}[t]
\centering\includegraphics[width=\textwidth]{figures/architektura-sieciowa}
\caption{Architektura sieciowa aplikacji}\label{rys:architektura-sieciowa}
\end{figure}

Ze względu na specyfikę aplikacji umożliwiającą edycję danych dotyczących danej hodowli koni wielu użytkownikom, w szczególności korzystających w większości z urządzeń mobilnych oraz sporadycznie z komputerów osobistych, zdecydowano się na formę systemu webowego z przeglądarkowym interfejsem użytkownika. Aplikacje internetowe pozwalają na wykorzystanie bogatego i dojrzałego zaplecza technologicznego do wyświetlania stron internetowych, co znacząco upraszcza aspekt dystrybucji oprogramowania oraz współbieżnego dostępu do danych.
Ze względu na specyfikę aplikacji umożliwiającą edycję danych dotyczących danej hodowli koni wielu użytkownikom, w szczególności korzystających w większości z urządzeń mobilnych oraz sporadycznie z komputerów osobistych, zdecydowano się na formę systemu webowego z przeglądarkowym interfejsem użytkownika. Aplikacje internetowe pozwalają na wykorzystanie bogatego i dojrzałego zaplecza technologicznego do wyświetlania stron internetowych, co znacząco upraszcza aspekt dystrybucji oprogramowania oraz współbieżnego dostępu do danych.

Dla zachowania kompatybilności z technologią progresywnych aplikacji internetowych oraz modularności zdecydowano się na rozdzielenie warstwy przetwarzania logiki biznesowej z dostępem do danych, z którą użytkownik nie wchodzi w bezpośrednią interakcję - nazywaną backend, od warstwy bezpośrednio obsługującej interfejs użytkownika i wizualną prezentację danych - frontend. Obie warstwy nie komunikują się ze sobą, lecz współdziałają w celu tworzenia pełnej funkcjonalności aplikacji. Użytkownik inicuje wszelkie interakcje z obydwoma komponentami, choć zapytania do backendu są inicjowane w przeglądarce użytkownika wchodzącego w interakcję z frontendem, a także otrzymane dane i komunikaty diagnostyczne są wyłącznie obsługiwane przez ten komponent. Komunikacja odbywa się wyłącznie na warstwie 7 modelu ISO/OSI (warstwie aplikacji) za pomocą protokołu HTTP. Frontend obługuje żądania zwracając pliki HTML, CSS i JavaScript, natomiast backend działa w bezstanowej architekturze sieciowego interfejsu programowania REST zwracającego dane w formacie serializacji JSON. 




\subsection{Frontend}

Korzystając z technologii progresywnych aplikacji internetowych (ang. Progressive Web App, PWA) oraz praktyki responsywnego projektowania stron internetowych (ang. responsive web design, RWD) zaimplementowano spójny interfejs użytkownika dla platform mobilnych. Do implementacji technologii progresywnych aplikacji internetowych wymagane jest, aby interfejs graficzny użytkownika był skomponowany w zdecydowanej większości ze statycznych dokumentów hipertekstowych HTML (HyperText Markup Language), arkuszy stylów CSS (Cascading Style Sheets) oraz kodu w języku programowania JavaScript. Statyczne pliki w podanym kontekście oznaczają podleganie nieczęstym zmianom poprzez pobranie zaktualizowanej wersji z serwera - wyłącznie między oddzielnymi wydaniami aplikacji i nigdy w celu wyświetlenia nowych danych. Interaktywna prezentacja danych jest możliwa natomiast poprzez zmianę struktury wyświetlanego dokumentu hipertekstowego wykonywaną po stronie klienta z użyciem kodu JavaScript w przeglądarce. Wymienione działanie jest wykonywane poprzez interfejs programowania Document Object Model (DOM, pl. \textit{Model Obiektowy Dokumentu}) pozwalający na częściową lub pełną modyfikację dowolnego węzła (np. paragrafu tekstu, separatora, kontrolek, obiektu multimedialnego) w hierachicznej strukturze drzewa dokumentu HTML. Interfejs ten jest w praktyce wykorzystywany przez bibliotekę React, która nieco modyfikuje model z dokumentów HTML i skryptów JavaScript na komponenty wielokrotnego użytku (zawierające wymagane fragmenty dokumentu HTML i funkcje JavaScript) budujące deklaratywne widoki (strony internetowe). Komponenty można powiązać z danymi, czyli zaprojektować ich wygląd w celu prezentacji danych zgodnie z ich schematem oraz skonfigurować ich źródło, które będzie automatycznie odświeżane dążąc do zadeklarowanego stanu. Komponenty można w sobie zagnieżdżać, odtwarzając te same aspekty hierachicznej struktury, a ich zmiany są śledzone poprzez użycie wewnętrznej dla tej biblioteki kopii Modelu Obiektowego Dokumentu nazywanego Virtual DOM (pl. Wirtualny Model Obiektowy Dokumentu).

Opisane podejście nazywa się Aplikacją jednostronicową (ang. \textit{Single-page application, SPA}), co wynika głównie ze sposobu aktualizowania interfejsu na podstawie zmian w danych poprzez kod po stronie klienta i przesyłania statycznych dokumentów HTML, pomijając jednocześnie aspekt użycia wielu stron. W przypadku prezentowanej pracy rama projektowa została dostarczona przez bibliotekę React Router, która dostarcza pakiet budujący do Aplikacji jednostronicowej, dostarczając ponadto zestaw narzędzi służących do przechodzenia pomiędzy osobnymi widokami w zależności od użytego adresu URL - nazwany router, a także moduły do efektywnego pobierania danych w tle za pomocą funkcji asynchronicznych.


\subsection{Backend}

Backend:
- integruje różne systemy:
    - zapis danych do bazy, a wyświetlanie ich na froncie w różnych formatach
    - obsługuje zapytania do klasyfikatora i LLM
    - wysyła okresowo powiadomienia email
    - obsługuje generowanie linków do przesyłania zdjęć na bucket
- obsługuje bezpośrednio:
    - uwierzytelnianie i kontrolę dostępu z użyciem Better Auth:
        - użytkownicy, hodowle (organizacje), sesje i hashe danych uwierzytelniających są zapisane w bazie danych z adapterem Drizzle
        - sesja przechowywana jest jako ciasteczko httpOnly 
    - logikę biznesową (CRUD, obliczenia ważności, )
    - współbieżny dostęp do danych
    - migracje i początkowe populacje bazy danych
- jest silnie typowany w Typescript, ze schematem w Zod oraz relacjami bazy danych z DrizzleORM
- można zapisać wszystkie jego endpointy do postaci OpenAPI, a także do klienta Hono RPC
- jest zasadniczo bezstanowy, chociaż nie natywnie klastrowalny ze względu na cron joby (by były powtórzenia powiadomień)
- napisany w Typescript, kompilowany do Javascript i uruchamiany w nodejs (distroless docker)
- parametry konfiguracyjne zależne od środowiska są konfigurowane przez zmienne środowiskowe w pliku .env i bibliotekę dotenv




Przetwarzanie wykonywane po stronie serwera jest realizowane w znacznej części przez komponent backendu. Został on zaimplementowany jako bezstanowy interfejs programowania REST, który obsługuje komunikację za pomocją danych serializowanych do formatu JSON. Integruje on między sobą wiele komponentów, co zostało pokazane na diagramie architektury sieciowej, Rysunek \ref{rys:architektura-sieciowa}. Jego głównym zadaniem jest udostępnienie dostępu do danych zapisanych w bazie danych przez protokół HTTP interfejsu sieciowego na warstwie 7 modelu ISO/OSI. Rola serwera HTTP jest spełniana przez framework \textit{Hono}. Framework ten pozwala na bardzo modularne tworzenie silnie typowanych endpointów, warstw pośrednich przetwarzania (\english{middleware}) i elastyczne łączenie ich w poszczególne ścieżki obsługiwane współbieżnie i wydajnie przez jeden z kilku dostępnych silników dopasowań nazywanych \english{Router}. Mimo obszernego wyboru w przypadku utworzonej aplikacji wykorzystywany jest wyłącznie najprostszy \textit{LinearRouter} ze względu na brak mierzalnych opóźnień przetwarzania wynikających z wyboru teoretycznie wolniejszego algorytmu dopasowań (pomiary były wyłącznie zależne od czasu komunikacji z bazą danych i innymi zewnętrzymi usługami) i pełną przewidywalność dopasowań, co w końcowej fazie projektu okazało się być problemem ze względu na złożoność ścieżek dopasowań. Ze względu na wymóg współbieżnego, spójnego dostępu do danych o złożonych, ale stabilnych relacjach została wybrana relacyjna baza danych \textit{PostgreSQL}. Spośród dostępnych na rynku rozwiązań wyróżnia się ona brakiem opłat licencyjnych, bogatą dokumentacją, bardzo dobrą optymalizacją przetwarzania na małych i średnich zbiorach danych, obszernym zestawem funkcji, a także jest dobrze znana i lubiana przez zespół projektowy. Zapytania do bazy danych są realizowane przez bibliotekę \textit{node-postgres}, która jest głównym silnikiem do tworzenia puli połączeń do bazy danych. Biblioteka ta jest rozszerzana przez bibliotekę \textit{DrizzleORM} w celu utworzenia Obiektowo Relacyjnego Mapowania struktur obiektów utworzonych w języku TypeScript na relacje bazy danych w języku SQL, napisania zoptymalizowanych zapytań do bazy danych, utworzenia migracji pomiędzy wersjami schematu tabel w procesie tworzenia aplikacji i jej wdrożenia, a także populowania bazy danych początkowymi danymi. Biblioteka \textit{DrizzleORM} charakteryzuje się zbliżoną semantyką, co język SQL, dzięki czemu dobra znajomość tworzenia zapytań przez zespół premiowała dobrą optymalizacją przepływu danych i prostszym, wydajniejszym modelem abstrakcji - każde poprawne zapytanie utworzone z użyciem \textit{DrizzleORM} jest konwertowane na dokładnie jedno zapytanie do bazy danych, a przygotowane wcześniej zapytania można było przepisać z małymi zmianami składni. Dzięki podejściu do połączenia z bazą danych z niewielką warstwą abstrakcji możliwe było wykorzystanie bardziej zaawansowanych funkcjonalności bazy danych PostgreSQL, takich jak m.in. użycie schematów jako dynamicznego zamiennika osobnych baz danych dla różnych, odseparowanych środowisk na tym samym serwerze, reprezentacja typów wyliczeniowych z użyciem \textit{pg\_enum} (m.in. do rodzajów koni i zdarzeń profilaktycznych), filtrowanie powiadomień do wysłania względem różnicy czasu wyliczanej po stronie silnika PostgreSQL, grupowanie kilku zapytań w jedno dzięki składni \textit{UNION} (powszechnie wykorzystywane przy pobieraniu zdarzeń profilaktycznych) i podzapytaniom z klauzulami \textit{NOT EXISTS} (do znalezienia czy istnieje domyślne zdjęcie dla danego konia przy usuwaniu), \textit{PARTITION BY} (do znalezienia najnowszych aktywnych zdarzeń profilaktycznych z każdej kategorii dla danego konia). \textit{DrizzleORM} jednocześnie dostarcza pełne interfejsy i typy danych oczekiwane przez język TypeScript, a nawet schematy walidacji do biblioteki walidacyjnej \textit{Zod}, dzięki dodatkowi \textit{drizzle-zod}. Pozwoliło to w procesie tworzenia aplikacji na zachowanie spójności w przepływie danych oraz zaoszczędzenie czasu na pisanie złożonych definicji schematów. Wykorzystanie biblioteki \textit{Zod} stanowi istotny element pracy, gdyż ze względu na zaawansowane i ergonomiczne funkcje walidacji, a także ogromną popularność biblioteka stała się standardem w środowisku TypeScript do tworzenia modeli danych. \textit{Zod} pozwala na zdefiniowanie schematów opisujących układ pól i ograniczenia danych, a następnie weryfikować wartości arbitralnych, niesprawdzonych obiektów przyjmowanych przez program, aby przetransformować lub potwierdzić poprawność z założonym formatem danych. Cenną funkcją biblioteki Zod jest zwracania szczegółowej, ustrukturyzowanej listy niezgodności ze schematem w trakcie walidacji. Zbiory powiązanych schematów można połączyć w rejestry, co pozwala powiązać je z dodatkowymi ustrukturyzowanymi metadanymi. Gotowe schematy można również obustronnie konwertować do formatu zgodnego ze specyfikacją JSON Schema, w tym OpenAPI. Wymieniona specyfikacja OpenAPI zapewnia spójny sposób przesyłania informacji na każdym etapie cyklu życia API. Jest to język specyfikacji dla interfejsów API HTTP, który definiuje strukturę i składnię w sposób niezalezny od języka programowania, w którym utworzono API. Ekosystem narzędzi powiązany z tym rozwiązaniem jest bardzo bogaty i dostarcza wiele rozwiązań generujących w pełni zautomatyzowany sposób gotową implementację (zdefiniowanej części sieciowej) klienta bądź serwera dla danej specyfikacji. Specyfikacje te są pisane w formatach YAML lub JSON, co umożliwia łatwe udostępnianie i ich wykorzystanie. Dzięki użyciu biblioteki \textit{@hono/zod-validator} możliwe było zdefiniowanie schematu danych przyjmowanych przez poszczególne endpointy API Hono z automatyczną walidacją niezależnie od formatu danych (query parameters, body jako json, czy form-data) z użyciem funkcji middleware \textit{zValidator}. Dodatkowo zestaw powiązanych bibliotek \textit{hono-openapi} i \textit{@hono/zod-openapi} pozwolił na dodanie adnotacji do dokumentacji danego endpointu w formacie OpenAPI oraz zdefiniowanie schematu zwracanych danych w zależności od kodu odpowiedzi. Wygenerowana dokumentacja w uniwersalnym formacie OpenAPI z całego backendu posłużyła jako wzorzec do tworzenia zapytań przez Gemini w procesie przetwarzania intencji użytkownika. Również dzięki interfejsowi graficznemu dostarczanemu przez bibliotekę \textit{Scalar} do obsługi dokumentacji OpenAPI niewielkim nakładem pracy utworzono zaawansowany panel administracyjny do zarządzania użytkownikami i stadninami przez administratorów. Schematy Zod wykorzystywane są również do walidacji konfiguracji zmiennych środowiskowych wymaganych do poprawnego działania aplikacji.

Aplikacja Moje konie korzysta ze scentralizowanego, zintegrowanego z kodem aplikacji systemu uwierzytelniającego. W celu zapewnienia bezpiecznego systemu uwierzytelniającego skorzystano z zewnętrznej biblioteki BetterAuth, która dostarczyła komponenty do zarządzania użytkownikami, kontami, hodowlami koni w kontekście organizacji i przynależnościami do nich. Biblioteka wykorzystuje PostgreSQL jako magazyn danych i tworzy automatycznie wymagane relacje poprzez integracje z DrizzleORM. Konta użytkowników mogą zostać tworzone wyłącznie przez użytkownika z grupy administratorów, nie jest możliwa rejestracja zewnętrzna bezpośrednio przez użytkownika. Uwierzytelnianie odbywa się wyłącznie za pomocą adresu e-mail i hasła, a adresy e-mail muszą zostać potwierdzone poprzez link. Użytkownicy mają możliwość zmiany ustawionego hasła, a po zakończeniu tego procesu następuje unieważnienie wszystkich aktywnych sesji, co powstrzymuje przed dostępem z potencjalnie przejętych urządzeń.

Zarządzanie sesjami realizowane jest poprzez mechanizm ciasteczek i zapisanie aktywnych sesji w bazie danych. W celu optymalizacji wydajności i redukcji obciążenia bazy przy sekwencyjnych żądaniach, włączono mechanizm zapisu ciasteczek w pamięci podręcznej serwera z czasem życia wynoszącym 5 minut. Sesje posiadają długi czas wygasania wynoszący 30 dni, w trakcie którego ważność sesji jest automatycznie odświeżana przy aktywności użytkownika raz na 24h. Ma to na celu utrzymywanie poprawnej sesji (stanu bycia zalogowanym do strony) na czas dostatecznie długi, aby nie stanowił problemu dla aktywnych i powracających po niedługich odstępach czasu użytkowników, a jego długość jest podyktowana zarządzaniem danymi o nieczęstej zmienności i niską potrzebą codziennego pobierania danych przez użytkowników. Pozostałe aspekty ciasteczek sesyjnych skonfigurowano z naciskiem na bezpieczeństwo. Flaga SameSite ustawiona na Strict wymusza pełną zgodność nazwy domenowej serwera z którym następuje połączenie w odniesieniu do serwera, który ustawił dane ciasteczko, co pomaga zapobiegać atakom polegającym na wymuszaniu nieautoryzowanej akcji w przeglądarce ofiary takim jak Cross Site Request Forgery (\akronim{CSRF}). Atrybut HttpOnly zabrania dostępu do wartości danego ciasteczka sesji przez kod JavaScript w przeglądarce klienta, co ponownie zawęża możliwe ataki z użyciem złośliwych skryptów takie jak Cross Site Scripting (\akronim{XSS}). Finalnie flagi Secure oraz Partitioned są skonfigurowane zgodnie z zaleceniami przeglądarek i wymuszają odpowiednio poufność przez konieczność stosowania protokołu HTTPS oraz izolację sesji użytkownika zgodnie wytycznymi prywatności.

Warstwa autoryzacji wykorzystuje model kontroli dostępu opartej na 2 rolach (admin, user) oraz strukturę wielodostępową (multi-tenancy) poprzez tworzenie ogranizacji do zarządzania odrębnymi hodowlami. Domyślna rola użytkownika pozwala na zarządzanie cyklem życia (dodawanie, usuwanie, edycja) wszystkich zwykłych obiektów takich jak: konie, wydarzenia z nimi związane, zapisani specjaliści, a także ustawienia użytkownika. Rola administratora jest przeznaczona dla osób posiadających odpowiednie kompetencje do zarządzania całą instancją serwera, a zatem pozawala na dodawanie nowych i edycję istniejących kont użytkowników, hodowli, danych uwierzytelniających i wszystkie uprawnienia nadane podstawoej roli użytkownika. Dodatkowo włączono obsługę kluczy API, są to jednoskładnikowe dane uwierzytelniające służące do obsługi żądań do API poprzez interfejsy programistyczne i może być wykorzystywane do automatyzacji części procesów aplikacji z zewnętrznymi usługami, a także posłużyć jako składnik uwierzytelniający w panelu administracyjnym. Pełen zestaw funkcji jest zawarty w dokumentacji w specyfikacji OpenAPI. Warstwa autoryzacji integruje się z serwerem utworzonym przez framework Hono poprzez wystawienie endpointów autoryzacyjnych pod ścieżką /api/auth. Dzięki wykorzystaniu gotowego zestawu tłumaczeń komunikaty błędów funkcji \textit{BetterAuth} zostały przetłumaczone na język polski. W celu poprawy widoczności informacji diagnostycznych o incydentach kluczowe zdarzenia bezpieczeństwa są rejestrowane przez dedykowany logger.

% tabele:
% - z aplikacji:
%     - zdjecia_koni
%     - choroby
%     - podkucia
%     - konie
%     - leczenia
%     - rozrody
%     - zdarzenia_profilaktycznego
%     - weterynarze
%     - kowale
%     - notifications
% - z better auth:
%     - organization
%     - invitation
%     - jwks
%     - verification
%     - member
%     - user
%     - session
%     - account
%     - apikey

\begin{figure}[t]
\centering\includegraphics[width=\textwidth]{figures/schemat-bazy.png}
\caption{Schemat bazy danych}\label{rys:schemat-bazy-danych}
\end{figure}

Aplikacja posiada bogaty model danych, co ma swoje odniesienie w złożonym modelu encji związków. W celu zachowania jak najlepszej spójności zastosowano znormalizowaną postać bazy danych, a dla ułatwienia edycji głównego identyfikatora sztuczne klucze główne. Aplikacja została zaprojektowana z myślą, aby obsługiwać wiele oddzielnych hodowli na jednej instancji aplikacji w modelu architektury \english{multi-tenant}. Ze względu na to założenie wszystkie obiekty dodawane przez użytkowników muszą mieć powiązanie z daną hodowlą, której nadany identyfikator rodziela przynależność do danego klienta -- dzierżawcy. Postawową funkcjonalnością aplikacji jest zarządzanie końmi, w tym przypisanie im imion, nr paszportowych, istotnych dat, a także hodowli do której należą i innych danych charakterystycznych. Dodatkowo dla każdego konia można przypisać dowolną liczbę zdjęć, których podstawowe dane o przynależności są zapisywane w bazie danych (do którego konia, czy dane zdjęcie jest domyślne), natomiast same pliki zapisywane w oddzielnej usłudze zoptymalizowanej do przechowywania plików binarnych Google Cloud Storage w danym zasobniku. Dla każdego konia można przypisać wiele wydarzeń takich jak podkucia, choroby, leczenia, rozrody czy zdarzenia profilaktyczne. W przypadku podkuć, leczeń, rozrodów i zdarzeń profilaktycznych przypisywany jest również specjalista -- kowal lub weterynarz, powiązany z danym wydarzeniem. W przypadku specjalistów przechowywane dane są proste i ograniczają się do wskazania imienia, nazwiska, numeru telefonu i hodowli w jakiej zostali dodani. Do podkuć, chorób i zdarzeń profilaktycznych przypisane sa daty zdarzenia/rozpoczęcia oraz ważności/zakończenia, a do wszystkich wydarzeń poza podkuciami również opisy. Zdarzenia profilaktyczne posiadają również pole wyboru rodzaju zdarzenia spośród: odrobaczania, podania suplementów, szczepienia, dentysty i inne. Natomiast wydarzenia -- rozrody dzielą się na: inseminację konia, sprawdzenie źrebności, wyźrebienie i inne. Istotna jest również możliwość powiązania leczeń do danej choroby konia. Utworzona bezpośrednio została również tabela notifications, która przechowuje preferencje użytkownika dotyczące powiadamiania o zliżającym się końcu ważności danego wydarzenia z konfigurowalną liczbą dni wyprzedzenia, godziną powiadomienia i metodą wysłania. Pozostałe tabele zostały utworzone przez bibliotekę BetterAuth i służą do zarządzania użytkownikami i ich przynależnością do danej hodowli (organizacji), uwierzytelniania i kontroli dostępu. Tabela organization przechowuje informacje o hodowli, w tym dostępną liczbą zapytań do wykorzystania przez moduł asystenta. Poprzez relację N -- N member jest powiązana z tabelą user, co pozwala odszukać przynależność użytkownika do danej hodowli, a także sprawdzić rolę konta (administrator, konto zwykłe). Tabele account i apikey przechowują dane uwierzytelniające dla danego użytkownika, głównie wykorzystywane jest hasło do uwierzytelniania zwykłych kont, natomiast klucz API do dostępu administracyjnego. W tabeli session odnajdziemy informacje o aktywnych sesjach logowania użytkowników. Tabela verification przechowuje informacje o wysłanych e-mailach z potwierdzeniem po założeniu konta. Natomiast tabela invitation nie jest w praktyce wykorzystywana w projekcie i dotyczy zaproszeń użytkowników do organizacji przez innych użytkowników aplikacji (w praktyce dane konto użytkownika jest wyłącznie dla jednej organizacji).




% Backend:
% - integruje różne systemy:
%     - zapis danych do bazy, a wyświetlanie ich na froncie w różnych formatach
%     - obsługuje zapytania do klasyfikatora i LLM
%     - wysyła okresowo powiadomienia email
%     - obsługuje generowanie linków do przesyłania zdjęć na bucket
% - obsługuje bezpośrednio:
%     - uwierzytelnianie i kontrolę dostępu z użyciem Better Auth:
%         - użytkownicy, hodowle (organizacje), sesje i hashe danych uwierzytelniających są zapisane w bazie danych z adapterem Drizzle
%         - sesja przechowywana jest jako ciasteczko sameSite: none,httpOnly: true, secure: true, partitioned: true, czyli niedostępne z poziomu kodu javascript po stronie klienta, następuje surowa walidacja czy nazwa domenowa nie uległa zmianie, wymusza protokół https i jest przechowywane osobno dla różnych kontekstów strony (wyłącznie dla api)
%         - wysyłanie potwierdzeń założenia konta, 
%     - logikę biznesową (CRUD, obliczenia ważności, generowanie raportów, przetwarzanie ustawień użytkowników)
%     - zadania okresowe CRON (wysyłanie powiadomień email o końcu ważności)
%     - współbieżny dostęp do danych z użyciem transakcji
%     - migracje i początkowe populacje bazy danych
% - jest silnie typowany w Typescript, ze schematem w Zod oraz relacjami bazy danych z DrizzleORM
% - można zapisać wszystkie jego endpointy do postaci OpenAPI, a także do klienta Hono RPC
% - jest zasadniczo bezstanowy, chociaż nie natywnie klastrowalny ze względu na cron joby (by były powtórzenia powiadomień)
% - napisany w Typescript, kompilowany do Javascript i uruchamiany w nodejs (distroless docker)
% - parametry konfiguracyjne zależne od środowiska są konfigurowane przez zmienne środowiskowe w pliku .env i bibliotekę dotenv
% - udostępnia statyczne pliki do klasyfikatora do załadowania testsetu
% - ma skonfigurowany jednolity logger requestów do API, zapytań do bazy danych, wydarzeń okresowych i innych operacji



% \subsection{Mechanizmy bezpiczeństwa}

% Traefik
% Docker
% Distroless / hardened images
% Cloudflare
% Firewall
% Ciasteczka?
% Better auth?
% 

% Bezpieczeństwo jest zapewniane przez protokół HTTPS poprzez wdrożenie aplikacji za reverse proxy, które jednocześnie centralizuje żądania do systemu i umożliwia niezależną skalowalność komponentów schowanych za nim. W przypadku tego projektu zdecydowano się na wybór Traefik, ze względu na wysoką wydajność przetwarzania zapytań, prostą konfigurację z dobrą integracją ze środowiskiem konteneryzacji Docker (w szczególności z Docker Compose) i zewnętrznym serwisem do wystawiania publicznie zaufanych certyfikatów TLS Let's Encrypt. 


\subsection{Wdrożenie i dystrybucja projektu}

% W github actions uruchamiany jest tsc i vite. Pliki *.js i definicja openapi jest kopiowana do obrazu backendu ...

\begin{figure}[t]
\centering\includegraphics[width=\textwidth]{figures/repozytorium-zaleznosci-budowanie.png}
\caption{Struktura repozytorium kodu, uproszczony diagram zależności oraz tworzenia paczki dystrybucyjnej}\label{rys:repozytorium-zaleznosci-budowanie}
\end{figure}

% Cały proces tworzenia paczki do dystrybucji dla części frontend jest złożony ze względu na optymalizację rozmiaru paczki, która to jest istotnym czynnikiem w szybkości ładowania strony, a także sztywny podział odpowiedzialności frontend - backend. Proces budowania przebiega następująco:
% \begin{enumerate}
% \item Kod w TypeScript z użyciem komponentów React w różnych plikach jest kompilowany do kodu w JavaScript (zmianę dodatkowych konstrukcji językowych na ich implementacje w JavaScript oraz pozbycie się typów), wyrażenia TailwindCSS zamieniane na klasy i dodawane do arkusza styli, a pliki multimedialne kopiowane bezpośrednio.
% \item Kod w JavaScript i style CSS są minifikowane (krótkie arbitralne nazwy funkcji i plików, usuwanie pomijalnych białych znaków i komentarzy, przepisywanie wyrażeń na krótsze o równoważnym znaczeniu).
% \item Zawartość plików jest przenoszona do innych plików o arbitralnych, unikalnych nazwach i stałym rozmiarze
% \item Zależności są analizowane pod kątem wykorzystania funkcji, a nieużywany kod usuwany
% \item Wynikowy katalog jest kopiowany do obrazu Dockera z web serwerem HTTP Nginx, który jest skonfigurowany do serwowania plików statycznych z maksymalnym cache oraz manifestem progresywnej aplikacji internetowej.
% \end{enumerate}


\begin{figure}[t]
\centering\includegraphics[width=\textwidth]{figures/diagram-wdrozenia.png}
\caption{Diagram wdrożenia z użyciem Terraform i Ansible}\label{rys:diagram-wdrozenia}
\end{figure}

