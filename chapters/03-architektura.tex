
\chapter{Architektura systemu}

Przyjęte ograniczenia:
- tani w utrzymaniu długoterminowym, najlepiej bezpłatny
- niewielki narzut czasowy na operacje podtrzymania, możliwie prosty
- wystarczająco wydajny, by zapewnić dostęp dla max 400 użytkowników (100 w danym momencie czasu) do tej samej bazy danych
- bezpieczny, bazujący na rozwiązaniach open-source i modularny, ułatwiając migrację i zmniejszając możliwy obszar nadużyć w przypadku naruszenia bezpieczeństwa
- oferuje przyjazny i intuicyjny interfejs graficzny do użycia w przeglądarce na komputerze oraz w postaci aplikacji mobilnej PWA

Ze względu na specyfikę aplikacji umożliwiającą edycję danych dotyczących danej hodowli koni wielu użytkownikom, w szczególności korzystających w większości z urządzeń mobilnych oraz sporadycznie z komputerów osobistych, zdecydowano się na formę systemu webowego z interfejsem użytkownika dostępnym w przeglądarce. Korzystając z technologii progresywnych aplikacji internetowych (ang. Progressive Web App, PWA) oraz praktyki responsywnego projektowania stron internetowych (ang. responsive web design, RWD) zaimplementowano spójny interfejs użytkownika dla platform mobilnych.  





Możliwa lista tematów:
- (INFRA) opis infrastruktury: konfiguracja ~sprzętowa vmki, dysków, bucketów z uzyciem terraform, przygotowanie zależności i ustawień systemu operacyjnego z ansible, modularność i powtarzalność środowiska uruchomieniowego z docker
- (NET) opis ruchu sieciowego end2end: dns i waf z cloudflare, reverse proxy traefik, firewall, podział frontend - backend, komunikacja backchannel z zewnętrznymi usługami (certy, mail, gemini), optymalizacja dostarczania z optimistic caching i pwa
- (GIT) opis działania potoków ciągłej integracji / ciągłego wdrażania z ansible i github actions, hostowanie repozytorium git monorepo, model współpracy nad kodem pull request i weryfikacja jakości kodu z eslint i etapem kompilacji 
- (DATA) opis modelu danych i sposobu ich przechowywania, schemat bazy danych, główne rodzaje zapytań, zapewnienia spójności danych, sposób backupu, współdzielony model walidacji danych zod
- (INNE) ciekawe zagadnienia architektury: modularność za pomocą mikro usług, współdzielenie schematu danych i dostępnych funkcji API za pomocą RPC, optymalizacja kosztów stałych do zera za wyjątkiem dns





Referencje:
- PWA:
  1. https://web.dev/explore/progressive-web-apps?hl=pl
  2. https://developer.mozilla.org/en-US/docs/Web/Progressive_web_apps
- RWD:
   1. https://developer.mozilla.org/en-US/docs/Learn_web_development/Core/CSS_layout/Responsive_Design