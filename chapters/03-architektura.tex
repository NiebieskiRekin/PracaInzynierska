
\chapter{Architektura systemu}

Możliwa lista tematów:
- (INFRA) opis infrastruktury: konfiguracja ~sprzętowa vmki, dysków, bucketów z uzyciem terraform, przygotowanie zależności i ustawień systemu operacyjnego z ansible, modularność i powtarzalność środowiska uruchomieniowego z docker
- (NET) opis ruchu sieciowego end2end: dns i waf z cloudflare, reverse proxy traefik, firewall, podział frontend - backend, komunikacja backchannel z zewnętrznymi usługami (certy, mail, gemini), optymalizacja dostarczania z optimistic caching i pwa
- (GIT) opis działania potoków ciągłej integracji / ciągłego wdrażania z ansible i github actions, hostowanie repozytorium git monorepo, model współpracy nad kodem pull request i weryfikacja jakości kodu z eslint i etapem kompilacji 
- (DATA) opis modelu danych i sposobu ich przechowywania, schemat bazy danych, główne rodzaje zapytań, zapewnienia spójności danych, sposób backupu, współdzielony model walidacji danych zod
- (INNE) ciekawe zagadnienia architektury: modularność za pomocą mikro usług, współdzielenie schematu danych i dostępnych funkcji API za pomocą RPC, optymalizacja kosztów stałych do zera za wyjątkiem dns






